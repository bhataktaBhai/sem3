\lecture{01}{Tue 01 Aug '23}{Introduction}

\section{The Course} \label{sec:intro}
Half or a little more of the course will cover probability.
We will have a quiz every alternate week, and a homework every alternate week.
Tentatively, one quiz and one homework will be dropped.

\subsection{Professor} \label{sec:professor}
\textbf{Name:} Manjunath Krishnapur \\
\textbf{Email:} \href{mailto:manju@math.iisc.ac.in}{manju@math.iisc.ac.in}

\subsection{References} \label{sec:references}
\begin{itemize}
    \item \href{https://minerva.it.manchester.ac.uk/~saralees/statbook3.pdf}
        {Probability and Statistics for Engineers and Scientists} by
        Sheldon~M.~Ross.
    \item \href{http://staff.ustc.edu.cn/~ynyang/2022/books/2.5.pdf}{Statistics}
        by Freedman, Pisani and Purves.
        This is very light on mathematics.
\end{itemize}
\subsection{Grading} \label{sec:grading}
\begin{enumerate}
    \item \textbf{Final:} 50\%
    \item \textbf{Midterm:} 20\%
    \item \textbf{Quizzes/Assignments:} 30\%
\end{enumerate}

Statistics is about making the best use of data to make reasonable decisions.
Statistics makes racism mathematical.


\section{Probability} \label{sec:prob}
\begin{definition}[Discrete Probability Space] \label{def:prob:discrete_space}
    A \emph{discrete probability space} is an ordered pair $(\Omega, p)$ where:
    \begin{enumerate}
        \item $\Omega$ is a finite or countable set.
        \item $p : \Omega \to [0, 1]$ such that $\sum_{\omega \in \Omega} p(\omega) = 1.$
    \end{enumerate}
    We define three more notions:
    \begin{enumerate}
        \item \textbf{Event:} any subset of $\Omega$.
        \item \textbf{Probability of an event:} for an event
            $A \subseteq \Omega$, $P(A) = \sum_{\omega \in A} p(\omega)$
            is the probability of $A$.
        \item \textbf{Random variable:} a function $X : \Omega \to \R$.
    \end{enumerate}
\end{definition}
\begin{notation}
    $p(\omega)$ is often denoted as $p_{\omega}$.
\end{notation}
