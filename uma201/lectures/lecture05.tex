\lecture{05}{}{}

\subsection{Infinite Sums} \label{sec:infinite_sums}
Say $\Omega = \set{\omega_1, \omega_2, \omega_3, \dots}$ is countable, and
$f : \Omega \to \R$.
We need to make sense of $\sum_{\omega \in \Omega} f(\omega)$.

One attempt is to define
\begin{proposition}
    \[
        \sum_{\omega \in \Omega} f(\omega) = \lim_{n \to \infty}
            \sum_{j=1}^{n} f(omega_{i}).
    \]
\end{proposition}
However, even if the limit exists, it may depend on the order of the terms,
which is not what we want.

In the special case of $f \geq 0$, the sequence of partial sums is increasing.
Thus it is either bounded and converges, or is unbounded and diverges to
$\infty$.

What about reordering?
\begin{theorem}
    Let \[
        \Omega = \set{\omega_{1}, \omega_{2}, \omega_{3}, \dots}
        = \set{\omega'_{1}, \omega'_{2}, \omega'_{3}, \dots}
    \] and $\lim_{n \to \infty} \sum_{j = 1}^{n} f(\omega_{j}) = L$,
    $\lim_{n \to \infty} \sum_{j = 1}^{n} f(\omega'_{j}) = L'$ (possibly
    $\infty$).
    Then $L = L'$.
\end{theorem}
\begin{proof}
    For any $n \in \N$, there exists some $n' \in \N$ such that \[
        \set{\omega_{1}, \omega_{2}, \dots, \omega_{n}}
        \subseteq \set{\omega'_{1}, \omega'_{2}, \dots, \omega'_{n'}}.
    \] Thus $S_{n} \leq S'_{n'}$ and in the limit, $L \leq L'$.

    Similarly, $L' \leq L$, giving $L = L'$.
\end{proof}

\begin{theorem}[Absolute Convergence] \label{thm:countable:absolute_sum}
    If $\Omega$ is countable and $f : \Omega \to \R$, then \[
        \lim_{n \to \infty} \sum_{j = 1}^{n} f(\omega_{j})
    \] exists and is independent of ordering,
    provided $\sum_{\omega \in \Omega} \abs{f(\omega)}$ is finite.
    We say that $f$ is ``absolutely summable''.
\end{theorem}
\vspace{5pt} \hrule

\begin{examples}
    \item ``Pick a number at random from $[0, 1]$.''
        This is beyond the scope of our theory at present, since $[0, 1]$ is
        uncountable.
    \item ``Pick a number uniformly at random from \N.''
        This is not possible.
\end{examples}
\vspace{5pt} \hrule

\subsection{Probability Rules} \label{sec:probability_rules}
\begin{definition}[Indicator Function] \label{def:prob:indicator_function}
    Let $A \subseteq \Omega$.
    The \emph{indicator function} of $A$ is $\ind{A} : \Omega \to \set{0, 1}$ 
    with \[
        \ind{A}(\omega) = [\omega \in A] = \begin{cases}
            1 & \text{if } \omega \in A, \\
            0 & \text{if } \omega \notin A.
        \end{cases}
    \]
\end{definition}
\begin{lemma}
    For any $A, B \subseteq \Omega$, $\ind{A \cup B} = \ind{A} + \ind{B}
    - \ind{A \cap B}$.
\end{lemma}
\begin{proof}
    Let $\omega \in \Omega$.
    We have four cases:
    \begin{enumerate}
        \item ($\omega \notin A$, $\omega \notin B$) \[
            \ind{A \cup B}(\omega) = 0 = 0 + 0 - 0 = \ind{A}(\omega)
            + \ind{B}(\omega) - \ind{A \cap B}(\omega).
        \]
        \item ($\omega \notin A$, $\omega \in B$) \[
            \ind{A \cup B}(\omega) = 1 = 0 + 1 - 0 = \ind{A}(\omega)
            + \ind{B}(\omega) - \ind{A \cap B}(\omega).
        \]
        \item ($\omega \in A$, $\omega \notin B$) \[
            \ind{A \cup B}(\omega) = 1 = 1 + 0 - 0 = \ind{A}(\omega)
            + \ind{B}(\omega) - \ind{A \cap B}(\omega).
        \]
        \item ($\omega \in A$, $\omega \in B$) \[
            \ind{A \cup B}(\omega) = 1 = 1 + 1 - 1 = \ind{A}(\omega)
            + \ind{B}(\omega) - \ind{A \cap B}(\omega). \qedhere
        \]
    \end{enumerate}
\end{proof}
\begin{theorem}
    For any probability space $(\Omega, p)$ and $A, B \subseteq \Omega$,
    we have
    \begin{enumerate}
        \item $\Pr(\O) = 0 \leq \Pr(A) \leq 1 = \Pr(\Omega)$.
        \item $\Pr(A^{c}) = 1 - \Pr(A)$.
        \item If $A \cap B = \O$ then $\Pr(A \cup B) = \Pr(A) + \Pr(B)$.
        \item $\Pr(A \cup B) = \Pr(A) + \Pr(B) - \Pr(A \cap B)$.
    \end{enumerate} 
\end{theorem}
\begin{proof}
    All probabilities must be positive as they are sums of nonnegative terms.
    (ii), (iii), and (iv) follow from the lemma.
    (i) follows from (ii).
\end{proof}
