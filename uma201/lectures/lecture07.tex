\lecture{07}{Tue 22 Aug '23}{}
\subsection{Principle of Inclusion \& Exclusion} \label{sec:pie}
Let $A_1, A_2, \ldots, A_n$ be events in a probability space.
Then \[
    \Pr(A_1 \cup A_2 \cup \cdots \cup A_n) = \sum_{j = 1}^{n} (-1)^{j + 1} S_j
\] where $S_j = \sum_{1 \leq i_1 < \dots < i_j \leq n} \Pr(A_{i_1} \cap \cdots \cap A_{i_j})$.

\begin{proof}
    Let $A_1, \dots, A_{n}$ be events.
    Let $\omega \in \Omega$.
    If $\omega \notin A_1 \cup \dots \cup A_{n}$,
    Then $\omega$ is not counted on either the left or the right side.

    If $\omega \in A_1 \cup \dots \cup A_{n}$,
    Then $\omega$ is counted once on the left side.
    Let $l = \size{\set{i \in [1..n] \mid \omega \in A_i}}$.

    Then $p_\omega$ is counted $\binom{l}{j}$ times in $S_j$.
    Thus $p_\omega$ is counted $\sum_{j = 1}^{n} (-1)^{j + 1} \binom{l}{j}$ times on the right side.
    \begin{align*}
        \sum_{j = 1}^{l} (-1)^{j + 1} \binom{l}{j}
            &= \sum_{j = 0}^{l} (-1)^{j + 1} \binom{l}{j} + 1 \\
            &= 1 - \sum_{j = 0}^{l} (-1)^{j} \binom{l}{j} \\
            &= 1 \qedhere
    \end{align*}
\end{proof}

\begin{example}
    Prof.~Trelawny guesses a deck of $n$ cards. \[
        \Omega = S_{n} \times S_{n}
    \] where $S_{n}$ is the set of permutations of $n$.
    Assuming Prof.~Trelawny is a lying bitch, \[
        p(\pi, \sigma) = \left(\frac{1}{n!}\right)^{2}.
    \] What is the probability that all guesses are wrong?
    That is, $\pi(j) \neq \sigma(j)$ for all $j \in n$. \[
        E = \set{(\pi, \sigma) \in \Omega \mid \pi(j) \neq \sigma(j)
        \,\forall j \in n}.
    \] We have that \[
        E^{c} = \set{(\pi, \sigma) \in \Omega \mid \exists j \in n (
            \pi(j) = \sigma(j))}
    \] is a union of events \[
        E^c_j = \set{(\pi, \sigma) \in \Omega \mid \pi(j) = \sigma(j)}.
    \] Thus \[
        \Pr(E^c) = S_1 - S_2 + \dots
    \] where \begin{align*}
        S_k &= \sum_{0 \leq i_1 < \dots < i_k < n}
            \Pr(A_{i_1} \cap \dots \cap A_{i_k}) \\
            &= \binom{n}{k} \cdot \frac{n! (n - k)!}{n!^2} \\
            &= \frac{1}{k!}
    \end{align*}
    Thus \begin{align*}
        \Pr(E^c) &= \frac{1}{1!} - \frac{1}{2!} + \dots - (-1)^{n} \frac{1}{n!} \\
        \implies \Pr(E) &= 1 - \Pr(E^c) \\
            &= \sum_{j = 0}^{n} (-1)^{j} \frac{1}{j!} \\
            &\approx \frac{1}{e} \text{ for large } n \\
            &\approx 0.37
    \end{align*}
\end{example}
\begin{problem}
    Show that the probability of exactly $k$ correct guesses is
    approximately $\frac{1}{e} \frac{1}{k!}$ for small $k$ and large $n$.
\end{problem}
\begin{example}
    The probability of fraudster Trelawney getting at least $10$ cards right is
    approx \[
        1 - \frac{1}{e} \left(1 + \frac{1}{2!} + \dots + \frac{1}{10!}\right)
        \approx 1 - \frac{1}{e} \cdot e = 0
    \]
\end{example}
\begin{problem}
    What is the probability that Trelawney gets at least $k$ cards right?
\end{problem}

\subsection{Bonferroni's Inequalities} \label{sec:bonferroni}
\begin{theorem}[Bonferroni's Inequalities] \label{thm:prob:bonferroni}
    Let $A_1, \dots, A_n$ be events in a probability space.
    Then \[
        S_1 - S_2 \leq \Pr(A_1 \cup \dots \cup A_n) \leq S_1.
    \] The right inequality is sometimes called the \emph{union bound}.
\end{theorem}
\begin{proof}
    Let $l = \size{\set{i \in [1..n] \mid \omega \in A_i}}$.
    Then $p_\omega$ is counted $\binom{l}{j}$ times in $S_j$.
    Thus $\omega$ is counted $\binom{l}{1} - \binom{l}{2}$ times on the left,
    $\ind{\Omega}$ in the middle, and $\binom{l}{1}$ times on the right.

    $\binom{l}{1} - \binom{l}{2} = 1$ for $l = 1, 2$ and $\leq 0$ otherwise.
\end{proof}
\begin{example}
    We throw $r$ distinguishable balls into $m$ labelled bins.
    What is the probability that some bin is empty?

    We let $\Omega = [1..m]^{r}$ with uniform probability.
    Let $A_i$ be the event that bin $i$ is empty.
    Then $P(A_{k}) = \frac{(m - 1)^{r}}{m^{r}}$ for all $k$.
    $P(A_{k} \cap A_{l}) = \frac{(m - 2)^{r}}{m^{r}}$ for all $k \neq l$.
    
    Then from the Bonferroni inequalities,
    \begin{equation*}
        m \cdot \frac{(m - 1)^r}{m^r} - \binom{m}{2} \cdot \frac{(m - 2)^r}{m^r}
            \leq \Pr(A_1 \cup \dots \cup A_m)
            \leq m \cdot \frac{(m - 1)^r}{m^r}
    \end{equation*}
\end{example}
