\lecture{09}{Tue 29 Aug '23}{Conditional Probability}

\section{Conditional Probability} \label{sec:conditional}

\begin{definition}[Conditional Probability] \label{def:conditional}
    Let $(\Omega, p)$ be a probability space and let $B$ be an event with $\Pr(B) > 0$.
    For any event $A$, we define \[
        \Pr(A \given B) = \frac{\Pr(A \cap B)}{\Pr(B)}
    \]
\end{definition}
\begin{remark} \leavevmode
    \begin{itemize}
        \item $\Pr(B \given B) = 1$.
        \item $\Pr(B^{c} \given B) = 0$.
        \item $\Pr(\;\cdot \given B)$ has the properties of probability.
            \begin{itemize}
                \item If $A_1, A_2$ are disjoint, then \[
                    \Pr(A_1 \cup A_2 \given B) = \Pr(A_1 \given B) + \Pr(A_2 \given B).
                \]
                \item $\Pr(A^c \given B) = 1 - \Pr(A \given B)$.
            \end{itemize}
    \end{itemize}
\end{remark}
\begin{remark}
    Events $A$ and $B$ are independent if and only if $\Pr(B) = 0$ or
    $\Pr(A \given B) = \Pr(A)$.
\end{remark}
\begin{examples}
    \item A shuffled deck of cards.
        $\Omega = S_{52}$, $p : \omega \mapsto \frac{1}{52!}$.
        
        Let $B$ be the event that the first card is a spade,
        let $A$ be the event that the second card is a diamond,
        and let $C$ be the event that the second card is a queen.
        We have \begin{align*}
            \Pr(A) &= \frac{1}{4}, \\
            \Pr(B) &= \frac{1}{4}, \\
            \Pr(C) &= \frac{1}{13}.
        \end{align*}
        For the intersections we have, \begin{align*}
            \Pr(A \cap B) &= \frac{13 \cdot 13 \cdot 50!}{52!} \\
                &= \frac{13}{4 \cdot 51}, \\
            \Pr(B \cap C) &= \frac{(12 \cdot 4 + 1 \cdot 3) \cdot 50!}{52!} \\
                &= \frac{1}{52}, \\
            \Pr(C \cap A) &= \frac{1}{52}.
        \end{align*}
        Now for the conditional probabilities, 
        \begin{alignat*}{2}
            \Pr(A \given B) &= \frac{13}{51} \quad&&> \Pr(A) \\
            \Pr(C \given B) &= \frac{1}{4} \quad&&= \Pr(C) \\
            \Pr(A \given C) &= \frac{1}{13} \quad&&= \Pr(A) \\
            \Pr(B \given C) &= \frac{1}{13} \quad&&= \Pr(B) \\
            \Pr(B \given A) &= \frac{13}{51} \quad&&> \Pr(B) \\
            \Pr(C \given A) &= \frac{1}{4} \quad&&= \Pr(C).
        \end{alignat*}
\end{examples}
\begin{notation}
    We notate $\Pr(A \given B \cap C)$ as $\Pr(A \given B, C)$.
\end{notation}
\begin{proposition}[Intersections] \label{thm:conditional:intersections}
    For any events $A_1, \dots, A_n$ provided
    $\Pr(A_1 \cap \ldots \cap A_{n -1}) \neq 0$, we have \[
        \Pr(A_1 \cap \ldots \cap A_n) = \Pr(A_1) \Pr(A_2 \given A_1) \dots \Pr(A_n \given A_1, \dots, A_{n - 1}).
    \]
\end{proposition}
\begin{proposition}[Law of Total Probability] \label{thm:conditional:total}
    Let $B_1, \dots, B_n$ be a partition of $\Omega$.
    Then for any event $A$, we have \[
        \Pr(A) = \sum_{i = 1}^{n} \Pr(B_i) \Pr(A \given B_i)
    \] where multiplication short-circuits to $0$.
\end{proposition}

\begin{theorem}[Bayes' Theorem] \label{thm:conditional:bayes}
    Let $B_1, \dots, B_n$ be a partition of $\Omega$.
    Then for any event $A$ with $\Pr(A) > 0$, we have \[
        \Pr(B_i \given A) = \frac{\Pr(B_i) \Pr(A \given B_i)}{\sum_{j = 1}^{n} \Pr(B_j) \Pr(A \given B_j)},
    \] where multiplication short-circuits to $0$.
\end{theorem}

\begin{example}
    Suppose there is a disease $X$ that affects $1$ in $10000$ people.
    There is a test $T$ for $X$ that is $99\%$ accurate.
    That is, $\Pr(T \given X) = 0.99 = \Pr(T^c \given X^c)$.

    A person gets tested and the test is positive.
    What is the chance they have the disease?
    \begin{align*}
        \Pr(X \given T) &= \frac{\Pr(X) \Pr(T \given X)}{\Pr(X) \Pr(T \given X)
                + \Pr(X^{c}) \Pr(T \given X^{c})} \\
            &= \frac{10^{-4} \cdot 0.99}{10^{-4} \cdot 0.99 + (1 - 10^{-4}) \cdot 0.01} \\
            &= \frac{0.99}{0.99 + 9999 \cdot 0.01} \\
            &= \frac{0.99}{100.98} \\
            &\approx 0.98\%.
    \end{align*}
\end{example}
