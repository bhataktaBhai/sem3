\lecture{02}{Thu 03 Aug '23}{}
\begin{examplelist}
    \item $\Omega = \set{1, 2, \dots, 6}$ with $p(\omega) = \frac{1}{6}$ for all
        $\omega \in \Omega$.

        $A = \set{1, 6}$ is an event, and
        $P(A) = \frac{1}{6} + \frac{1}{6} = \frac{1}{3}$.

        This is a potential ``model'' for a fair d6.
    \item ``Toss $n$ coins.''
        We propose a probability space to describe this situation.
        Suppose $n = 2$.
        We label the coins and represent the set of all possible outcomes as \[
            \Omega = \set{HH, HT, TH, TT} \text{ or as }
            \Omega = \set{2H, 1H1T, 2T}.
        \]
        We choose the first representation since it carries the most
        information.

        $\Omega = \set{0, 1}^{n}$.
        We propose $p(\omega)$ to be $\frac{1}{2^{n}}$ for all $\omega$.
        Indeed, $\sum_{\omega \in \Omega} p_{\omega} = 1$.

        Let $A$ be the event that we get $k \in \set{0, 1, \dots, n}$ or more
        heads.
        That is, \[
            A = \set{x = (x_{1}, \dots, x_{n}) \in \Omega \mid \sum x \geq k}
        \] and $P(A) = \sum_{a \in A} p_{a} = \frac{1}{2^{n}} \abs{A}$.
        
        To compute $\abs{A}$, we note that $A = B_{k} \cup B_{k+1} \cup \dots
        \cup B_{n}$ where $B_{j} =
        \set{\omega \in \Omega \mid \sum \omega = k}$ are pairwise disjoint.
        Thus, $\abs{A} = \sum_{j = k}^{n} \abs{B_{j}}$ and so
        \begin{align*}
            P(A) &= \frac{\abs{B_{k}} + \dots + \abs{B_{n}}}{2^{n}} \\
                &= \frac{1}{2^{n}} \brk{
                    \binom{n}{k} + \binom{n}{k+1} + \dots + \binom{n}{n}
                }
        \end{align*}
    \item ``Toss a coin $n$ times.''
        $\Omega = \set{0, 1}^{n}$ again.
        The index $i$ in an element of $\Omega$ is whether the $i$th toss landed
        heads or tails.
        In the previous example, the index $i$ was whether the $i$th coin
        landed heads or tails.

        Is $p$ the same as before?
        If the coin has no ``memory'', then yes.
        But suppose the coin is made of clay.
        The shape of the coin changes as we toss it.
        Then, the probability of getting heads on the $i$th toss depends on
        the previous tosses.
        The coin has ``memory''.
        
        Of course, we don't usually bother with such details.
        It is still important to note that we are assuming that the coin has no
        memory.
    \item ``Throw $r$ balls into $m$ bins at random.''
        This situation describes several others.
        \begin{enumerate}
            \item Toss $r$ coins ($m = 2$).
            \item Roll an $m$-sided dice $r$ times.
            \item Record the birthdays of $r$ people ($m = 365$ or $366$).
            \item Study the occurence of letters in an English text.
                Here $m = 26$ and $r$ is the number of letters in the text.
        \end{enumerate}
        We propose $\Omega = \set{1, \dots, m}^{r}$.
        ``Random'' is not descriptive enough.

        We have $\abs{\Omega} = m^{r}$ and so a reasonable $p$ is
        $\omega \mapsto \frac{1}{m^{r}}$.

        An alternative is $\tilde{\Omega} =
        \set{\omega \in \set{0, \dots, r}^{m} \mid \sum \omega = r}$.
        Note that $\abs{\tilde{\Omega}} = \binom{m + r - 1}{m - 1}$
        A $\tilde{p}$ corresponding to the above $p$ is complicated.
        In choosing $\Omega$, we made an implicit assumption that the $r$ balls
        are distinguishable, as are the $m$ bins.

        Suppose we have $r$ indistinguishable balls and $m$ distinguishable
        bins.
        Is $p = \omega \mapsto \frac{1}{\abs{\tilde{\Omega}}}$ reasonable?
\end{examplelist}
