\documentclass[12pt]{article}
\input{~/IISc/preamble}
\usepackage{algorithm}
\usepackage{algpseudocodex}

\usepackage{algorithm}
\usepackage{algpseudocodex}

\usepackage{algorithm}
\usepackage{algpseudocodex}

\usepackage{siunitx}

\title{Assignment 01}
\author{Naman Mishra}
\date{04 August, 2023}

\begin{document}
\maketitle

\begin{notation}
    \begin{align*}
        [a..b] &\coloneqq [a, b] \cap \Z \\
        [P] &\coloneqq \begin{cases}
            1 & \text{if $P$ is true} \\
            0 & \text{if $P$ is false}
        \end{cases} \\
        S_{n} &\coloneqq \set{\text{permutations of } [1..n]}
    \end{align*}
\end{notation}
\setcounter{section}{1}
\begin{problem}
    A fair coin is tossed $n$ times.
    Show that the probability that there is exactly one run of heads and two
    runs of tails is $\frac{(n - 1)(n - 2)}{2^{n + 1}}$.
    [\textit{Note}: A run is a consecutive sequence of tosses giving the same
    result.
    For example, $001101000$ has three runs of tails and two runs of heads]
\end{problem}
\begin{solution}
    We model the probability space as $\Omega = \{0, 1\}^n$ with $p_{\omega}$
    being $\frac{1}{\size{\Omega}} = \frac{1}{2^{n}}$ for all $\omega$.
    This is because the coin is fair, and each toss is assumed to be independent
    of the others.

    Let $A$ be the event that there is exactly one run of heads and two runs
    of tails.
    This implies that the first head is at an index greater than $1$ and the
    last head at an index less than $n$ ($1$-indexed).
    Thus we can write $A$ as \[
        A = \set{\omega \in \Omega \mid \exists x < y \in [2..n]
            \text{ such that } \omega_{j} = [x \leq j < y]}.
    \]
    Let $B$ be the set of subsets of $[2..n]$ of size $2$.
    Let $f : B \to A$ be given as $f(b) = a$ where \[
        a_{j} = [\min b \leq j < \max b] \text{ for all } j.
    \]
    Then $f$ is a bijection.

    Thus we have $\size{A} = \size{B} = \binom{n-1}{2}$ and so the probability
    of $A$ is \begin{align*}
        \Pr(A) &= \sum_{\omega \in A} p_{\omega} \\
            &= \frac{\size{A}}{2^{n}} \\
            &= \frac{(n - 1)(n - 2)}{2^{n + 1}}
    \end{align*} as required.
\end{solution}

% \begin{lemma}
%     Let $S$ be a set with $N$ elements.
%     Let $k \in [1..N]$ and let $m \in [0..N]^{k}$ such that $\sum m \leq N$.
%     Let \begin{multline*}
%         B = \{b \in \mathscr{P}(S)^{k} \mid \forall i \in [1..k], \size{b_{i}} = m_{i} \\
%         \text{ and } \forall i, j \in [1..k], i \neq j \implies b_{i} \cap b_{j} = \O \}.
%     \end{multline*}
%     Then \[
%         \size{B} = \binom{N}{m_{1}} \binom{N - m_{1}}{m_{2}} \cdots
%         \binom{N - \sum_{i=1}^{k-1} m_{i}}{m_{k}}.
%     \]
% \end{lemma}
\begin{problem}
    A well-shuffled deck of cards is dealt among four players
    (so $13$ cards each).
    What is the chance that no player has two cards of the same kind
    (\textit{i.e.}, two aces or two ones, or two queens, etc)?
    Find an exact answer, and also give an approximate numerical value.
\end{problem}
\begin{solution}
    We assign a number from $1$ to $4$ to each player,
    a number from $1$ to $4$ to each suit,
    and a number from $1$ to $13$ to each face.

    We can thus model each card as $(f, s) \in C = [1..13] \times [1..4]$, where
    $f$ is the face and $s$ is the suit.
    
    We model the sample space as \[
        \Omega = \set{
            Q \in M_{13 \times 4}([1..4]) \mid
            f(p, Q) = 13 \text{ for all } p \in [1..4]
        },
    \] where $f(x, A)$ is the number of occurences of $x$ in $A$, and
    $Q_{ij}$ represents the player to whom the card $(i, j)$ is dealt.

%     We introduce a natural total order $\leq$ on $C$ as \[
%         (f, s) \leq (\tilde{f}, \tilde{s}) \iff
%         4f + s \leq 4\tilde{f} + \tilde{s}.
%     \] Let \[
%         \Omega_{1} = \set{
%             X \in (C^{13})^{4} \mid
%             Q_{ij} = k \iff (i, j) \in X_{k}
%         }.
%     \] Let \[
%         \Omega_{2} = \set{
% k
%         }
%     \]
    We can find $\size{\Omega}$ as follows: \\
    First we choose $13$ out of the $52$ positions in the matrix to be $1$. \\
    Then we choose $13$ out of the remaining $39$ positions to be $2$. \\
    Then we choose $13$ out of the remaining $26$ positions to be $3$. \\
    Finally, we choose $13$ out of the remaining $13$ positions to be $4$.

    This gives a total of \[
        \binom{52}{13} \binom{39}{13} \binom{26}{13} \binom{13}{13}
    \] possible ways to form an element of $\Omega$.

    Thus $\size{\Omega} =
    \binom{52}{13} \binom{39}{13} \binom{26}{13} \binom{13}{13}
    = \frac{52!}{13!^{4}}$.

    Since the deck is well-shuffled, it is fair to assume that each element of
    $\Omega$ is equally likely.
    Thus $p_{\omega} = \frac{1}{\size{\omega}} = \frac{13!^{4}}{52!}$ for all
    $\omega \in \Omega$.

    We can model the event as \[
        E = \set{
            \omega \in \Omega \mid
            \forall i \in [1..13] (\forall j \in [1..4] \forall k \in [1..4] (
            j \neq k \implies \omega_{ij} \neq \omega_{ik}
            ))
        }
    \] and so by the pigeonhole principle, \[
        E = \set{
            \omega \in \Omega \mid
            \forall i \in [1..13]
            \left(\set{\omega_{ij}}_{j \in [1..4]} = [1..4]\right)
        }
    \] and thus \[
        E = \set{
            \omega \in \Omega \mid
            \forall i \in [1..13]
            \left((\omega_{ij})_{j \in [1..4]} \in S_{4}\right)
        }.
    \] We can define a bijection $\phi : S_{4}^{13} \to E$ by \[
        \phi(\varsigma)_{ij} = (\varsigma_{i})_{j}
    \] and so $\size{E} = \size{S_{4}^{13}} = 4!^{13}$.

    Therefore, \[
        \Pr(E) = \sum_{\omega \in E} p_{\omega} = \frac{4!^{13} 13!^{4}}{52!}.
    \] This is approximately equal to $1.63 \cdot 10^{-11}$.
    
    If interpreted as seconds, it is roughly the time it takes light to travel a
    distance of $\SI{5}{\milli\meter}$.
\end{solution}
% \begin{problem}
%     
% \end{problem}
% \begin{solution} \leavevmode
%     \begin{enumerate}
%         \item We can model the sample space as $\Omega = [1..n]^{k}$ with
%             $p_{\omega} = \frac{1}{\size{\Omega}} = \frac{1}{n^{k}}$ for all
%             $\omega \in \Omega$.
%
%             The event that $1$ is not in the chosen sample is $A = [2..n]^{k}$.
%
%             Thus $\Pr(A) = \sum_{a \in A} p_{a} = \frac{\size{A}}{\size{\Omega}}
%             = \paren{1-\frac{1}{n}}^{k}$
%
%         \item We can model the sample space as \[
%             \Omega = \set{\omega \in \mathscr{P}([1..n]) \mid \size{\omega} = k}
%         \] with $p_{\omega} = \frac{1}{\size{\Omega}} = \binom{n}{k}^{-1}$
%         for all $\omega \in \Omega$.
%
%         The event that $1$ is not in the chosen sample is \begin{align*}
%             A &= \set{\omega \in \Omega \mid 1 \notin \omega} \\
%               &= \set{\omega \in \mathscr{P}([1..n]) \mid \size{\omega} = k
%                 \text{ and } 1 \notin \omega} \\
%               &= \set{\omega \in \mathscr{P}([2..n]) \mid \size{\omega} = k}.
%         \end{align*}
%         Thus \[
%             \Pr(A) = \sum_{a \in A} p_{a}
%                    = \frac{\size{A}}{\size{\Omega}}
%                    = \frac{\binom{n-1}{k}}{\binom{n}{k}}
%                    = \frac{n - k}{n}.
%         \]
%     \end{enumerate}
% \end{solution}
%
% \begin{problem}
%     A die is thrown repeatedly till the face $6$ comes up for the first time.
%     Write the probability space and calculate the probability of the event that
%     the face $1$ never showed up.
% \end{problem}
% \begin{problem}
%     A fair die is thrown, and if the number $k$ shows up, a fair coin is tossed
%     $k$ times.
%     Write down the probability space fully and calculate the probability of the
%     event that no heads show up.    
% \end{problem}
% \begin{problem}
%     Generalize the first problem to compute the probability that in a sequence
%     of $n$ tosses of a fair coin, there are a total of $k$ runs (note that if
%     there are $j$ runs of heads, the number of runs of tails is either $j - 1$
%     or $j$ or $j + 1$).
% \end{problem}
% \begin{solution}
%     We model the sample space as $\Omega = \set{0, 1}^{n}$ with
%     $p_{\omega} = \frac{1}{\size{\Omega}} = \frac{1}{2^{n}}$ for all
%     $\omega \in \Omega$.
%
%     If there are $k$ runs, then the sequence toggles between $0$ and $1$ $k - 1$
%     times.
%     Thus we can model the event as \begin{multline*}
%         A = \left\{\omega \in \Omega \mid \exists x_{1} < \dots < x_{k-1} \in 
%         [2..n] \right.\\
%         \left.\text{ such that }
%         \omega_{j} = \left.\begin{cases}
%             1 - \omega_{j-1} & \text{if } j \in \set{x_{i}} \\
%             \omega_{j - 1} & \text{otherwise}
%         \end{cases}\right\}\right\}
%     \end{multline*}
% \end{solution}

\end{document}
