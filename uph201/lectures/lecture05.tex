\lecture{05}{Wed 16 Aug '23}{}

\subsection{First Law of Thermodynamics} \label{sec:FLOT}
Defines \emph{internal energy}.

When $\delta q$ is the heat supplied to the system, and $\delta w$ is the work
done by the system, then the first law of thermodynamics states that the change
in internal energy $U$ is given by \[
    \dd U = \delta q + \delta w.
\] This is a restatement of the conservation of energy, more generalised than in
mechanics.

How do we measure $\delta q$ and $\delta w$?

\subsubsection{Heat} \label{sec:heat}
\emph{Specific heat} ($C$) is the energy required to change the temperature of
a unit mass of a substance by one kelvin.

\emph{Molar specific heat} ($C_{\textrm{mol}}$) is the energy required to change
the temperature of a mole of a substance by one kelvin.
\[
    \delta q = m C \Delta T.
\] Since heat (and work) is path dependent, we need to specify more than just
the initial and final states of the system.
We must also specify the path taken by the system.

Two common paths are \emph{isobaric} and \emph{isochoric}.
The corresponding specific heats are \emph{specific heat at constant pressure}
($C_p$) and \emph{specific heat at constant volume} ($C_v$).

In case of solids and liquids, $C_p \approx C_v$.

We can however generalise the specific heat at constant volume to
\emph{anergetic specific heat} ($C_{\textrm{an}}$).
This is along a path where no work is done by the system.
In the particular case of an ideal gas, this path is the same as the isochoric
one. \\
\hrule
Not all heating processes lead to an increase in temperature.
For example, heating a block of ice at $\SI{-10}{C}$ will look
like:
\begin{center}
    \begin{tikzpicture}
        \begin{axis}[
                xlabel=$t$,
                ylabel=$T$,
                % xmin=-10,
                % xmax=300,
                % ymin=-10,
                % ymax=110,
                % axis lines=middle,
                % xtick={-10, 0, 100},
                % ytick={0, 100},
                xticklabels={},
                % yticklabels={$0$, $100$},
                % legend pos=north west,
            ]
            \addplot[domain=-10:0, color=red] {x};
            \addplot[domain=0:20, color=green] {0};
            \addplot[domain=20:120, color=red] {x-20};
            \addplot[domain=120:170, color=green] {100};
            \addplot[domain=170:220, color=red] {x-70};
            % \legend{$\delta q$, $\delta w$, $\dd U$}
        \end{axis}
    \end{tikzpicture}
\end{center}
\hrule

\subsubsection{Work} \label{sec:work}
Work done by an ideal gas is defined as \[
    \delta w = p \dd V.
\] Is this $p$ the internal pressure or the external pressure?

For \emph{any} process from state $1$ to $2$, \[
    U_{2} - U_{1} = Q - W,
\] irrespective of the path taken.

But what does the ``path'' mean here?
\begin{center}
    \begin{tikzpicture}
        \draw (0, 0) node[left]{1}
            .. controls (2, 1) and (2, -2) ..
            (3, -2) node[right]{2};
    \end{tikzpicture}
\end{center}
Every point along a curve describing a thermodynamic process is an equilibrium
state.
This is called a \emph{quasi-static} process.
Such a process must take infinite time to complete.

In such a process, the internal and external pressures are always equal.
