\lecture{01}{Fri 04 Aug '23}{Introduction}

\section{The Course} \label{sec:course}
\subsection{Schedule} \label{sec:schedule}
\begin{itemize}
    \item MWF 12:00-13:00 lecture hours (no tutorial on Monday)
    \item Tutorials will still be held occasionally with homework discussions
        and quizzes.
\end{itemize}
\subsection{Syllabus} \label{sec:syllabus}
\begin{center}
    \begin{tikzpicture}
        \node (classical) [process] {Classical Physics};
        \node (thermo) [process, below=of classical] {Thermo and stat physics};
        \node (quantum) [process, below=of thermo] {Quantum physics};
        \node (special) [process, right=of quantum] {Special};
        \node (general) [process, right=of special] {General};
        \path (general) -- (special) node[midway] (A) {};
        \node [process] (relativity) at (A |- thermo) {Relativity};
        \path (quantum) -- (special) node[midway] (B) {};
        \node (standard) [process, below=of B] {Standard model};

        \path [arrow] (classical) -- (thermo);
        \path [arrow] (thermo) -- (quantum);
        \path [arrow] (classical) -| (relativity);
        \path [arrow] (relativity) -- (special);
        \path [arrow] (relativity) -- (general);
        \path [arrow] (quantum) -| (standard);
        \path [arrow] (special) -| (standard);
    \end{tikzpicture}
\end{center}
\subsection{Grading} \label{sec:grading}
\begin{itemize}
    \item \textbf{Quizzes:} $20\%$.
    \item \textbf{Midterm:} $40\%$.
    \item \textbf{Final:} $40\%$.
\end{itemize}

\section{Classical Physics} \label{sec:classical}
\subsection{Newton's Laws} \label{sec:newton}
\begin{enumerate}[label=(\Roman*)]
    \item In an inertial frame, $\bm{F}_{\mathrm{net}} = 0 \implies \bm{v} =
        \mathrm{const}$.
    \item $\ddot{\bm{r}} = \frac{\bm{F}_{\mathrm{net}}}{m}$ where $m$ is the
        inertial mass.
    \item For any $2$ particles, $\bm{F}_{12} = -\bm{F}_{21}$.
\end{enumerate}

\subsubsection{Math: Vector Equations} \label{sec:vector_equations}
\begin{equation*}
    \ddot{\bm{r}} = \frac{\bm{F}}{m} \equiv \begin{cases}
        \ddot{x} = \frac{F_{x}}{m} \\
        \ddot{y} = \frac{F_{y}}{m} \\
        \ddot{z} = \frac{F_{z}}{m}
    \end{cases}
\end{equation*}
For a system of $N$ particles, we have $3N$ equations of motion.
With $6N$ initial conditions, the evolution of the system is uniquely
determined.

\subsection{Conservation Laws} \label{sec:conservation_laws}
\subsubsection{Momentum} \label{sec:conservation:momentum}
\subsubsection{Energy} \label{sec:conservation:energy}
\subsubsection{Angular Momentum} \label{sec:conservation:angular_momentum}
