\lecture{03}{Wed 9 Aug '23}{}

Every path is a deviation from the classical path. \begin{align*}
    q(t) &= q_{cl}(t) + \delta q(t) \\
    \text{with } \delta q(t_{1}) &= \delta q(t_{2}) = 0
\end{align*}
Let $q_{0}(t)$ be the optimal solution, \textit{i.e.}, $\pdv{S}{q} \equiv 0$.
\begin{align*}
    q(t) &= q_{0}(t) + \delta q(t) \\
    \dot{q}(t) &= \dot{q}_{0}(t) + \delta \dot{q}(t) \\
    \delta S &= \int_{t_{1}}^{t_{2}}
        L(t, q(t), \dot{q}(t)) - L(t, q_{0}(t), \dot{q}_{0} (t)) \dd x
\end{align*}
For small $\delta q$, \textit{i.e.},
$\delta q(t) \ll q(t) \forall t \in [t_{1}, t_{2}]$,
\begin{align*}
    \delta S &= \int_{t_{1}}^{t_{2}} \pdv{L}{q}(t, q_{0}(t), \dot{q}_{0}(t)) \dd x
\end{align*}
% \begin{align*}
%     \int_{t_{1}}^{t_{2}} \pdv{L(q, \dot{q}, t)}{q} \dd t &= 0
% \end{align*}
% We derive \[
%     \boxed{\dv{}{t}\left(\pdv{L}{\dot{q}}\right) - \pdv{L}{q} = 0}
% \]
\begin{examples}
    \item \textbf{Free particle:} $L = \frac{1}{2} m \dot{q}^{2}$, so
        \begin{align*}
            \dv{}{t}\left(\pdv{L}{\dot{q}}\right) &= 0 \\
            \dv{}{t}{}(m \dot{q}) &= 0 \\ % TODO: wtf
            \ddot{q} &= 0.
        \end{align*}
    \item $L = \frac{1}{2} m \dot{q}^{2} - V(q)$.
        \begin{align*}
            \dv{}{t}\left(\pdv{L}{\dot{q}}\right) - \pdv{L}{q} &= 0 \\
            \dv{}{t}(m \dot{q}) + \pdv{V}{q} &= 0 \\
            m \ddot{q} &= - \pdv{V}{q}.
        \end{align*}
\end{examples}

In general, for a Lagrangian specified by
$L(q_{1}, \dots, q_{f}, \dot{q}_{1}, \dots, \dot{q}_{f}, t)$, \[
    \dv{}{t}\left(\pdv{L}{\dot{q}_{i}}\right) - \pdv{L}{q_{i}} = 0 \text{ for } i = 1, \dots, f.
\] If $L$ does not depend on $q_{j}$ \emph{explicitly} for some $j$, then \[
    \dv{}{t} \pdv{L}{\dot{q}_{j}} \equiv 0
\] Such a coordinate is called a \textbf{cyclical} or \textbf{ignorable coordinate}.

This is a generalised way of understanding conservation laws.
\begin{itemize}
    \item $\theta$-independent Lagrangians have rotational symmetry, so
        angular momentum is conserved.
    \item $x$-independent Lagrangians have translational symmetry, so linear
        momentum is conserved.
    \item $t$-independent Lagrangians have time symmetry, so energy is conserved.
\end{itemize}
