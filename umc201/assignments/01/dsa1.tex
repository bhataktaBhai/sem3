\documentclass{article}
\usepackage{amsmath, amssymb, amsthm, mathtools,}
\usepackage[ruled, lined, longend]{algorithm2e} 
\title{DSA Assignment 1}
\author{Ishaq Hamza (22187), Nagasai (), Naman Mishra ()}
\date{UMC201}
\begin{document}
\maketitle
\section*{Question 2.6}
\textbf{Notation:} $A[a_1,..., a_2][b_1,...,b_2]$ denotes the following subset of the array $A[m][n]$
$$\{A[i][j] ~ \vert a_1 \leq i \leq a_2, ~ j = 1,...,n\} \cup \{A[i][j] ~ \vert b_1 \leq i \leq b_2, ~ i = 1,...,m\}$$
\subsection*{Part a}
\begin{algorithm}
    \KwIn{an $n \times n$ array $A[n][n]$ of integers which is sorted row-wise and column-wise, integer $x$}
    \KwOut{index $(i, j)$ of the element $x$ if present or NULL if absent in the array}
    \For{$i = 1$ \KwTo $n$}{
        t = binarySearch($A[i][1,...,n]$, $x$)\; 
        \If{$t \neq -1$}{
            return $(i, t)$\;
        }
    }
    return NULL\;
    \caption{$\mathcal{O}(n \log n)$ algorithm to search for an element in a sorted row-wise and column-wise array}
\end{algorithm}
Here the binarySearch function is the same as we discussed in class.\\
\textbf{Proof of Correctness}\\
\textbf{Loop Invariant:} $\mathsf{L}(i): x \notin A[1,...,i-1][1,...,n]$\\
\textbf{Base Case:} $i = 1$\\
$$\mathsf{L}(1): x \notin A[1,...0][1,...,n]$$
the above holds trivially as the array $A[1,...,0][1,...,n]$ is empty.\\
\textbf{Inductive Step:} assume that $\mathsf{L}(i)$ holds before entering the loop\\
$$x \notin A[1,...,i-1][1,...,n]$$
\textbf{Case 1:} $x \notin A[i][1,...,n]$\\
    \begin{align*}
        &\implies x \notin A[1,...,i-1][1,...,n] \cup A[i][1,...,n]\\
        &\implies \mathsf{L}(i+1)\\
    \end{align*}
    $i \mapsto i+1$, therefore $L$ holds after the iteration of the loop\\
\textbf{Case 2:} $x \in A[i][1,...,n]$\\
binarySearch returns the index of $x$ in $A[i][1,...,n]$, therefore the algorithm returns the index $(i, t)$ and the loop breaks, hence $\mathsf{L}$ holds vacuously\\
\\
Therefore the loop invariant holds inductively.\\
\\
\textbf{Termination:} the loop terminates when $i = n+1$\\
$\mathsf{L}(n+1)$: $x \notin A[1,...,n][1,...,n]$ (which is the complete array)\\
$\implies x \notin A$, output should be NULL.
$$\mathbb{Q} \mathbb{E} \mathbb{D} $$
\textbf{Time Complexity:} the loop runs for $n$ iterations and each iteration performs a binary search on an array of size $n$, hence the time complexity is $\mathcal{O}(n \log n)$\\
\newpage
\subsection*{Part b}
\begin{algorithm}
    \KwIn{an $m \times n$ array $A[m][n]$ of integers which is sorted row-wise and column-wise, integer $x$}
    \KwOut{index $(i, j)$ of the element $x$ if present or NULL if absent in the array}
    $i = m$, $j = 1$\;
    \While{$i > 0$ and $j < n+1$}{
        \If{$A[i][j] < x$}{
            $i--$\;
        }
        \ElseIf{$A[i][j] < x$}{
            $j++$\;
        }
        \Else{
            return $(i, j)$\;
        }
    }
    \caption{$\mathcal{O}(m + n)$ algorithm to search for an element in a sorted row-wise and column-wise array}
\end{algorithm}
\textbf{Proof of Correctness}\\

we define the loop invariant $\mathsf{L}$ as follows:\\
$$\mathsf{L}(i, j): x \notin A[i+1,...,m][1,...,j-1]$$
we shall prove the invariance inductively\\
\textbf{Base Case:} $i = m, j = 1$\\
$$\mathsf{L}(m, 1): x \notin A[m+1,...,m][1,...,0]$$
the above statement holds trivially as in the array $A[m+1,...,m][1,..,0]$ is empty\\\\
\textbf{Inductive Step:} assume that $\mathsf{L}(i, j)$ and the loop condition hold before entering the loop\\
$$x \notin A[i+1,...,m][1,...,j-1]$$
\textbf{Case 1:} $A[i][j] > x$\\
    \begin{align*}
        &\implies x < A[i][j] \leq A[i][j+1] \leq ... \leq A[i][n]\\
        &\implies x \notin A[i][j,j+1...,n]\\
    \end{align*}
    this result when combined with the induction hypothesis yeilds
    $$x \notin A[i+1,...,m][1,...,j-1] \cup A[i][j,j+1,...,n] ~ (= A[i,...,m][1,...,j])$$
    $$\implies \mathsf{L}(i+1, j)$$
    $i \mapsto i+1$, therefore $L$ holds after the iteration of the loop\\
\textbf{Case 2:} $A[i][j] < x$\\
    \begin{align*}
        &\implies x > A[i][j] \geq A[i-1][j] \geq ... \geq A[1][j]\\
        &\implies x \notin A[1,2,...,i-1][j]\\
    \end{align*}
    this result when combined with the induction hypothesis yeilds
    $$x \notin A[i+1,...,m][1,...,j-1] \cup A[1,2,...,i-1][j] ~ (= A[1,...,i][j,...,n])$$
    $$\implies \mathsf{L}(i, j+1)$$
    $j \mapsto j+1$, therefore $L$ holds after the iteration of the loop\\
\textbf{Case 3:} if $A[i][j] = x$, then the algorithm returns the index and the loop breaks, hence $\mathsf{L}$ holds vacuously\\
\\
Therefore the loop invariant holds inductively.\\
\\
\textbf{Termination:} the loop terminates when either $i = 0$ or $j = n+1$\\
\textbf{Case 1:} i = 0\\
$\mathsf{L}(0, j)$: $x \notin A[1,...,m][1,...,j-1]$ (which is the complete array)\\
$\implies x \notin A$, output should be NULL.\\
\textbf{Case 2:} j = n+1\\
$\mathsf{L}(i, n+1)$: $x \notin A[i+1,...,m][1,...,n]$ (which is the complete array)\\
$\implies x \notin A$, output should be NULL.
$$\mathbb{Q} \mathbb{E} \mathbb{D} $$
\textbf{Time Complexity:} the loop runs for at most $m+n$ iterations and each iteration performs a constant number of comaprisons and updates, hence the time complexity is $\mathcal{O}(m+n)$\\\\
\\
\section*{Question 2.7}
first we define the following function $t-Asum$
\newpage
\begin{algorithm}
    \KwIn{an array $A[1,...,n]$ of integers and an integer $t$ such that $1 \leq t \leq n$}
    \KwOut{an array $A'[1,...,n-t+1]$ such that $A'[i] = A[i]+...+A[i+t-1]$}
    $A'[1] = A[1]+...+A[t]$\;
    \For{$i = 2$ to $n-t+1$}{
        $A'[i] = A'[i-1] - A[i-1] + A[i+t-1]$\;
    }
    return $A'$\;
    \caption{$\varTheta(n)$ algorithm to compute the array $A'$} 
\end{algorithm}
\textbf{Proof of Correctness}\\
\textbf{Loop Invariant:} $\mathsf{L}(i): A'[i] = A[i]+...+A[i+t-1]$\\
\textbf{Base Case:} $i = 1$\\
$$\mathsf{L}(1): A'[1] = A[1]+...+A[t]$$
the above holds trivially as $A'[1] = A[1]+...+A[t]$\\
\textbf{Inductive Step:} assume that $\mathsf{L}(i)$ holds before entering the loop\\
$$A'[i] = A[i]+...+A[i+t-1]$$
\begin{align*}
    A'[i+1] &= A'[i] - A[i] + A[i+t]\\
    &= A[i]+...+A[i+t-1] - A[i] + A[i+t]\\
    &= A[i+1]+...+A[i+t]\\
\end{align*}
$i \mapsto i+1$, therefore $L$ holds after the iteration of the loop\\
\\
Therefore the loop invariant holds inductively.\\
\\
\textbf{Termination:} the loop terminates when $i = n-t+1$\\
this is guaranteed since $i$ is incremented by 1 in each iteration and the loop condition is $i \leq n-t+1$\\
\textbf{Time Complexity:} the loop runs for $n-t+1$ iterations and each iteration performs a constant number of additions and subtractions, hence the time complexity is $\varTheta(n)$\\\\
\subsection*{Part a}
\begin{algorithm}
    \KwIn{two arrays of integers $A[1,...,n]$ and $B[1,...,n]$}
    \KwOut{$(i, j, t)$ where $A[i]+...+A[i+t-1] = B[j],...,B[j+t-1]$ if such subarrays exist, otherwise returns the special value NIL}
    \For{$t = 1$ to $n$}{
        \For{$i = n$ to $n-t+1$}{
            $A' = t-Asum(A)$
            $B' = t-Asum(B)$
        }
        \For{$i = 1$ to $n$}{
            \For{$j = 1$ to $n$}{
                \If{$A'[i] = B'[j]$}{
                    return $(i, j, t)$\;
                }
            }
        } 
    }
    return NIL\;
    \caption{$\varTheta(n^3)$ algorithm to find $t$ consecutive elements in one array whose sum is the same as the sum of $t$ consecutive elements in the other array}
\end{algorithm}
\textbf{Time Complexity:} in the $t^{th}$ iteration of the outermost loop, first the array is made by calling the $t-Asum$ which is $\varTheta(n)$.
\begin{align*}
    \sum_{t=1}^{n}\varTheta(n) &= \varTheta(n^2)\\
\end{align*}                                                                                                                                         
the inner two loops comapre each pair of elements $(A'[i], B'[j])$ of the loops, hence the total number of comparisons is $\binom{n-t+1}{2}$.\\
\begin{align*}
    \sum_{t=1}^{n}\binom{n-t+1}{2} &= \sum_{t=1}^{n}\frac{(n-t+1)(n-t)}{2}\\
    &= \frac{1}{2}\sum_{t=1}^{n}(n^2 - 2nt + t^2 - n + t)\\
    &= \frac{1}{2}(\sum_{t=1}^{n}t^2 - (2n + 1)\sum_{t=1}^{n} + \sum{t=1}{n}n(n+1))\\
    &= \frac{1}{2}(\frac{n(n+1)(2n+1)}{6} - (2n+1)\frac{n(n+1)}{2} + n^2(n+1))\\
    &= \frac{(n-1)(n)(n+1)}{6}\\
    \frac{n^3}{6} < & ~ \frac{(n-1)(n)(n+1)}{6} < \frac{n^3}{3} ~ \forall n>1\\
\end{align*}
hence the time complexity is $\varTheta(n^3)$\\\\
\newpage
\subsection*{Part b}
\begin{algorithm}
    \KwIn{two arrays of integers $A[1,...,n]$ and $B[1,...,n]$}
    \KwOut{$(i, j, t)$ where $A[i]+...+A[i+t-1] = B[j],...,B[j+t-1]$ if such subarrays exist, otherwise returns the special value NIL}
    \For{$t = 1$ to $n$}{
        $A' = t-Asum(A)$
        $B' = t-Asum(B)$
        $sA'$ = sort($A'$)\;
        $sB'$ = sort($B'$)\;
        i = 1, j = 1\;
        \While{$i \leq n-t$ and $j \leq n-t$}{
            \If{$sA'[i] > sB'[j]$}{
                $j++$\;
            }
            \ElseIf{$sA'[i] < sB'[j]$}{
                $i++$\;
            }
            \Else{
                return $(i, j, t)$\;
            }
        }
    }
    return NIL\;
    \caption{$\varTheta(n^2 \log(n))$ algorithm to find $t$ consecutive elements in one array whose sum is the same as the sum of $t$ consecutive elements in the other array}
\end{algorithm}
Here, the sort function is assumed to work in $\varTheta(n \log n)$ time.\\
\textbf{Proof of Correctness}\\
we define the loop invariant $\mathsf{L}$ as follows:\\
$$\mathsf{L}(i, j): A'[a] \neq B'[b] ~ \forall a < i, ~ b < j$$
we shall prove the invariance inductively\\
\textbf{Base Case:} $i = 1, j = 1$\\
$$\mathsf{L}(1, 1): A'[a] \neq B'[b] ~ \forall a < 1, ~ b < 1$$
the above statement holds trivially as the array $A'[a] \neq B'[b] ~ \forall a < 1, ~ b < 1$ is empty\\\\
\textbf{Maintenance:} assume that $\mathsf{L}(i, j)$ and the loop condition hold before entering the loop\\
$$A'[a] \neq B'[b] ~ \forall a < i, ~ b < j$$
\textbf{Case 1:} $A'[i] > B'[j]$\\
    $$A'[a] \neq B'[b] ~ \forall a < i, ~ b < j+1$$
    $$\implies \mathsf{L}(i, j+1)$$
    $j \mapsto j+1$, therefore $L$ holds after the iteration of the loop\\
\textbf{Case 2:} $A'[i] < B'[j]$\\
    $$A'[a] \neq B'[b] ~ \forall a < i, ~ b < j+1$$
    $$\implies \mathsf{L}(i+1, j)$$
    $i \mapsto i+1$, therefore $L$ holds after the iteration of the loop\\
\textbf{Case 3:} if $A'[i] = B'[j]$, then the algorithm returns the index and the loop breaks, hence $\mathsf{L}$ holds vacuously\\
\\
Therefore the loop invariant holds inductively.\\
\\
\textbf{Termination:} the loop terminates when either $i = n-t+2$ or $j = n-t+2$\\
since at least one of $i$ or $j$ is incremented in each iteration, the loop terminates in at most $2(n-t+1)$ iterations.\\
\textbf{Case 1:} $i = n-t+2$\\
$\mathsf{L}(n-t+2, j)$: $A'[a] \neq B'[b] ~ \forall a < n-t+2, ~ b < j$\\
note that in this case, $i$ was incremented to $n-t+2$ only when $A'[n-t+1] < B'[j] (\leq B[j+1] \leq...\leq n-t+1)$, therefore $A'[n-t+1] \neq B'[b] ~ \forall b \leq n$ (since array is sorted)\\
$\implies A'[a] \neq B'[b] ~ \forall a \leq n-t+1, ~ b \leq n-t+1$\\
Hence we must return NIL.\\
\textbf{Case 2:} $j = n-t+2$\\
$\mathsf{L}(i, n-t+2)$: $A'[a] \neq B'[b] ~ \forall a < i, ~ b < n-t+2$\\
note that in this case, $j$ was incremented to $n-t+2$ only when $A'[i] > B'[n-t+1] (\geq B[n-t] \geq...\geq 1)$, therefore $A'[i] \neq B'[b] ~ \forall b \geq 1$ (since array is sorted)\\
$\implies A'[a] \neq B'[b] ~ \forall a \leq n-t+1, ~ b \leq n-t+1$\\
Hence we must return NIL.\\
$$\mathbb{Q} \mathbb{E} \mathbb{D}$$
\textbf{Time Complexity:} for each $t$, two calls to sort are made which are $\varTheta(n \log(n))$, subsequently $t-Asum$ is called, which is $\varTheta(n)$\\
$t$ ranges from $1$ to $n$, hence the overall time complexity of the algorithm is $\varTheta(n^2 \log(n))$\\  
\end{document}
