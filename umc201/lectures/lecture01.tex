\lecture{01}{Tue 8 Aug '23}{A Warning}
\section{The Course} \label{sec:course}
\subsection{Instructor} \label{sec:instructor}
Prof.~C.~Pandu Rangan. \\
\textbf{Office hours:} CDS 406, 2 pm onwards, Wednesday.

\subsection{Grading} \label{sec:grading}
Harsh.
Cluster-based absolute grading.

Tentative weightage for the theory component:
\begin{itemize}
    \item \textbf{Assignments:} $30\%$.
    \item \textbf{Midsem:} $20\%$.
    \item \textbf{Endsem:} $50\%$.
\end{itemize}

\subsection{References} \label{sec:references}
\begin{enumerate}
    \item \href{https://dl.ebooksworld.ir/books/Introduction.to.Algorithms.4th.Leiserson.Stein.Rivest.Cormen.MIT.Press.9780262046305.EBooksWorld.ir.pdf}
        {\textsl{Introduction to Algorithms}}
        by Cormen, Leiserson, Rivest, Stein. \\
        Good for pseudocode.
        Read fully over multiple semesters.
    \item \href{https://jeffe.cs.illinois.edu/teaching/algorithms/}
        {\textsl{Algorithms}}
        by Jeff Erickson. \\
        Great problem sets.
    \item \textsl{Algorithm Design} by Kleinberg and Tardos. \\
        Chatty style.
        Applied problems.
\end{enumerate}

\section{Analysis} \label{sec:analysis}
% Z-Specification Language
\begin{problem}[Binary Search] \label{prob:binary_search}
    Given an array $A$ sorted by $\leq$ and a key $k$, return the index of $k$
    in $A$ if it exists, else return $-1$.
\end{problem}
\begin{solution} \leavevmode
    \begin{algorithm}
        \begin{algorithmic}[1]
            \Function{BinarySearch}{$A[0..n-1]$, $k$}
                \State $p \gets 0$
                \State $q \gets n - 1$
                \While{$p \leq q$}
                    \State $m \gets \floor{(p + q) / 2}$
                    \If{$x < A[m]$}
                        \State $q \gets m - 1$
                    \ElsIf{$x > A[m]$}
                        \State $p \gets m + 1$
                    \Else
                        \State \Return $m$
                    \EndIf
                \EndWhile
                \State \Return $-1$
            \EndFunction
        \end{algorithmic}
    \end{algorithm}

    \begin{proof}
        [Proof of partial correctness.]
        Let $C = p \leq q$ and $I = x \in A[0..n-1] \implies x \in A[p..q]$.

        $I$ is trivially true before the loop.
        $C$ is false after the loop.

        Suppose $C \land I$ is true at the beginning of the loop for some
        instance.
        \begin{itemize}
            \item If $x = k$, then $p$ and $q$ are not modified, and so
                $C \land I$ remains true.
                The program terminates with a correct result.
            \item If $x < A[m]$, then $x < A[m..n-1]$ since $A$ is sorted.
                Thus if $x \in A$, then $x \in A[p..q] \setminus A[m..n-1]$ and
                so $x \in A[p..m-1]$.
                Thus $q \gets m-1$ preserves $I$.
            \item If $x > A[m]$, then $x > A[0..m]$ since $A$ is sorted.
                Thus if $x \in A$, then $x \in A[p..q] \setminus A[0..m]$ and
                so $x \in A[m+1..q]$.
                Thus $p \gets m+1$ preserves $I$. \qedhere
        \end{itemize}
    \end{proof}
    \begin{proof}
        [Proof of termination.]
        Consider two cases:
        \begin{enumerate}
            \item[($x \in A$)] By $I$, $x \in A[p..q]$ is always true.
                Thus $C \equiv p \leq q$ is always true.
                Thus $q - p$ is always a non-negative integer.

                Let $p_{j}, q_{j}$ be the values of $p, q$ at the beginning of
                the $j$-th iteration.
                For the $(j + 1)$-th iteration, we have that if $x = A[m]$, the
                procedure terminates.
                Otherwise,
                \begin{align*}
                    q_{j + 1} - p_{j + 1} &= \begin{cases}
                        \floor{\frac{p_{j} + q_{j}}{2}} - 1 - p_{j} & x < A[m] \\
                        q_{j} - \floor{\frac{p_{j} + q_{j}}{2}} - 1 & x > A[m]
                    \end{cases} \\
                        &\leq q_{j} - p_{j} - 1 \\
                        &< q_{j} - p_{j}.
                \end{align*}
                Thus the procedure must terminate.
            \item[($x \notin A$)] \qedhere
        \end{enumerate}
    \end{proof}
    \begin{proof}
        [Alternative proof of termination.]
        Let $T = q - p + 1$.
        $T$ is a non-negative integer that decreases in each iteration of the
        loop.
        Thus the loop cannot run indefinitely.
        By $C$ we have that $p \leq q$ at the beginning of each iteration.
        \begin{align*}
            T_{j} &= q_{j} - p_{j} + 1 \\
            T_{j + 1} &= q_{j + 1} - p_{j + 1} + 1 \\
                &= \begin{cases}
                    \floor{\frac{p_{j} + q_{j}}{2}} - p_{j} & x < A[m] \\
                    q_{j} - \floor{\frac{p_{j} + q_{j}}{2}} & x > A[m]
                \end{cases} \\
                &\leq q_{j} - p_{j} \\
                &< T_{j}. \qedhere
        \end{align*}
    \end{proof}

    \textbf{Complexity:}
    We have that for the worst case,
    $T(1) = 1$ and $T(n) = 1 + T(\floor{n/2})$ for $n > 1$.

    Claim: $T(n) = \log_{2}(n) + 1$.
    \begin{proof}
        $P(n)$ be that $T(m) = \log_{2}(m) + 1$ for all $m \leq n$.
        Let $Q(n) = P(2^{n})$.

        $Q(0)$ is trivially true.
        // Induct.
    \end{proof}
\end{solution}

\begin{solution}[Alternative] \leavevmode
    \begin{algorithm}
        \begin{algorithmic}[1]
            \Function{BinarySearch}{$A[0..n-1]$, $k$}
                \State $p \gets 0$
                \State $q \gets n - 1$
                \While{$p < q$}
                    \State $m \gets \floor{(p + q) / 2}$
                    \If{$x \leq A[m]$}
                        \State $q \gets m$
                    \Else
                        \State $p \gets m + 1$
                    \EndIf
                \EndWhile
                \If{$x = A[p]$}
                    \State \Return $p$
                \Else
                    \State \Return $-1$
                \EndIf
            \EndFunction
        \end{algorithmic}
    \end{algorithm}
\end{solution}

\begin{problem}[Buggy Binary] \label{prb:analysis:buggy_binary}
    Prove the incorrectness of the following algorithm for binary search.
    \begin{algorithmic}[1]
        \Function{BinarySearch}{$A[0..n-1]$, $k$}
            \State $p \gets 0$
            \State $q \gets n - 1$
            \While{$p < q$}
                \State $m \gets \ceil{(p + q) / 2}$ \label{line:buggy:ceil}
                \If{$x \leq A[m]$}
                    \State $q \gets m$
                \Else
                    \State $p \gets m + 1$
                \EndIf
            \EndWhile
            \If{$x = A[p]$}
                \State \Return $p$
            \Else
                \State \Return $-1$
            \EndIf
        \EndFunction
    \end{algorithmic}
    Fix the algorithm without modifying line \ref{line:buggy:ceil}.
\end{problem}
