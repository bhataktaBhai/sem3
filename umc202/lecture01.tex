\lecture{01}{Wed 02 Aug '23}{Introduction}

\section{The Course} \label{sec:course}
\subsection{Grading} \label{sec:grading}
Absolute grading.
$90 \pm 2$ marks out of $100$ for an A+.
\begin{itemize}
    \item \textbf{Final exam:} $50\%$.
    \item \textbf{Midterm exam:} $30\%$.
    \item \textbf{Assignments:} $20\%$.
\end{itemize}
\subsection{References} \label{sec:references}
\begin{itemize}
    \item \href{https://faculty.ksu.edu.sa/sites/default/files/numerical_analysis_9th.pdf}
        {Numerical Analysis} by Richard L. Burden and J. Douglas Faires
    \item 
\end{itemize}
\section{Introduction} \label{sec:intro}
Solving \emph{algebraic} systems of equations numerically.
\begin{examplelist}
    \item $F(x) = 0$.
    \item $F_{j}(x_{1}, \dots, x_{n}) = 0$ where $j \in \set{1, \dots, m}$.
    \item $y' = f(t, y)$ with $y(t_{0}) = y_{0}$.
    \item $y'' + a y' + b y = 0$ with either
        $y(t_{0}) = y_{0}, y'(t_{0}) = y_{1}$ or
        $y(t_{0}) = y_{0}, y(t_{1}) = y_{1}$.
\end{examplelist}

We'll do interpolation, root-finding techniques, differential equations with
initial conditions, etc.

\section{Single Variable Root-Finding} \label{sec:root-finding}
Given a continuous function $F : \R \to \R$, we want to find $x$ such that
$F(x) = 0$.

Let $x^{*}$ be a solution to $F(x) = 0$.
We algorithmically generate a sequence $\set{x_{n}}$ that tends to $x^{*}$.

\textbf{The algorithm:} Find two points $a$ and $b$ such that $F(a) F(b) < 0$.
By the intermidiate value theorem, there exists $x^{*} \in (a, b)$ such that
$F(x^{*}) = 0$.
We can perform a binary search to close in on $x^{*}$.

\begin{remark}
    This only works if such points $a$ and $b$ exist.
    The graph of $F$ could be tangent to the $x$-axis, as in $x \mapsto x^{2}$.
\end{remark}

\subsection{Fixed point} \label{sec:fixed_point}
We can rewrite $F(x) = 0$ as $x = g(x)$, where $g(x) = x + F(x)$.
Finding a root of $F$ is equivalent to finding a fixed point of $g$.
